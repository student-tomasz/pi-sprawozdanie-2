\documentclass[10pt,a4paper]{article}
\usepackage[a4paper]{geometry}

\usepackage{polski}
\usepackage{xltxtra}
\usepackage{indentfirst}
\usepackage{relsize}
\usepackage{fancyvrb}
\usepackage{hyperref}
\hypersetup{
    pdftitle={Sprawozdanie z ćwiczenia nr 2 z laboratorium Programowanie internetowe},%
    pdfauthor={Tomasz Cudziło},%
    colorlinks=true,        % false: boxed links; true: colored links
    linkcolor=black,        % color of internal links
    citecolor=green,        % color of links to bibliography
    filecolor=magenta,      % color of file links
    urlcolor=cyan,          % color of external links
    unicode=true,           % non-Latin characters in Acrobat’s bookmarks
    pdfstartview={FitH},    % fits the width of the page to the window
    pdfnewwindow=true       % links in new window
}
\usepackage{paralist}

%% tweak fonts
\defaultfontfeatures{Mapping=tex-text}
\setromanfont{Charis SIL}
\setsansfont[Scale=MatchLowercase]{Helvetica Neue}
\setmonofont[Scale=MatchLowercase]{Menlo}
\linespread{1.25}

%% define custom commands and environments
\DefineVerbatimEnvironment%
  {SmallVerbatim}%
  {Verbatim}{fontsize=\relsize{-0.5},numbers=left,numbersep=-10pt,frame=lines,tabsize=4}

\newcommand{\f}[1]{\texttt{#1}}
\newcommand{\s}[1]{\textsf{#1}}

\begin{document}

%%fakesection{Tytuł}
\title{
  Sprawozdanie z~ćwiczenia nr~2\\z~laboratorium Programowanie internetowe
}
\author{
  Tomasz Cudziło\\
  \textsc{PW EE Informatyka}\\[10pt]
}
\date{\today}
\maketitle



\section{Opis projektu}

Celem projektu było wykonanie menu składającego się z ikon reagujących na
zdarzenia kursora myszy.

\subsection{Kod źródłowy}
\begin{description}
  \item[Adres projektu:] \hfill \\
  \url{http://volt.iem.pw.edu.pl/~cudzilot/pi/cw2/}
  \item[Repozytorium projektu:] \hfill \\
  \url{https://github.com/student-tomasz/pi-cwiczenie-2}
  \item[Repozytorium sprawozdania:] \hfill \\
  \url{https://github.com/student-tomasz/pi-sprawozdanie-2}
\end{description}
Rewizja w czasie pisania sprawozdania:
\href{https://github.com/student-tomasz/pi-cwiczenie-2/tree/b3f54b187e99c161f7a6780524f504c9fd1b00dd}{\f{b3f54b1}}.

\subsection{Biblioteki}

Animacja ikon nie korzysta z zewnętrznych bibliotek. Strona projektu używa style
normalizujące z \f{YUI}. Wyświetlanie źródeł na stronie potrzebuje bibliotek
\f{jQuery} i \f{google-prettify}. Identycznie jak w projekcie pierwszym.

\subsection{Wykonanie}
Strona oraz jej style tj.
\href{https://github.com/student-tomasz/pi-cwiczenie-2/tree/b3f54b187e99c161f7a6780524f504c9fd1b00dd/index.html}{\f{index.html}}
i
\href{https://github.com/student-tomasz/pi-cwiczenie-2/tree/b3f54b187e99c161f7a6780524f504c9fd1b00dd/stylesheets/fancy-toolbar.css}{\f{stylesheets/fancy-toolbar.css}}
nie wpływają na zachowanie ikon. Logika animacji jest zawarta w
\href{https://github.com/student-tomasz/pi-cwiczenie-2/tree/b3f54b187e99c161f7a6780524f504c9fd1b00dd/javascripts/fancy-toolbar.js}{\f{javascripts/fancy-toolbar.js}}.

\subsubsection{Statyczne \f{HTML} i \f{CSS}}
Menu jest zaimplementowane jako lista nieuporządkowana \f{ul\#fancy-toolbar}.
Elementy przeznaczone do animacji są rozpoznawane jako \f{li.ft-item}. Elementy
te zawierają tylko jednego potomka --- obraz \f{img.ft-icon}.

Dla ułatwienia zależności w pozycjonowaniu, elementy listy mają nadpisany sposób
prezentacji na \f{display: inline-block;}. Są też wyłączone z normalnego układu
dokumentu przez \f{float: left;}, przez co ułożone są poziomo. Dodatkowo każdy
element ma przypisany \f{margin-top: 64px;}, co wyrównuje obrazy na wspólnej
linii u dołu listy. Elementy i obrazy są ostylowane jak do stanu gdy kursor
myszki znajduje się poza obrębem listy.

Sama lista ma stale aktualizowaną szerokość i jest wycentrowana w swoim przodku.

\subsubsection{Animacja z \f{JavaScript}}

Funkcja zajmująca się animacją ikon jest zawarta w pliku
\href{https://github.com/student-tomasz/pi-cwiczenie-2/tree/b3f54b187e99c161f7a6780524f504c9fd1b00dd/javascripts/fancy-toolbar.js}{\f{javascripts/fancy-toolbar.js}}.

Jako argument przyjmuje \f{id} elementu mającego pełnić rolę menu. Na podstawie
\f{id} wyszukuje w dokumencie węzeł reprezentujący menu. Przechodzi przez jego
potomków i zapamiętuje przynależące do klasy \f{ft-item}, są to elementy
przeznaczone do animacji. Każdemu z nich przypisywany jest \f{index}, początkowy
rozmiar \f{width} oraz funkcje obsługujące zdarzenia.

Obsługiwane są trzy typy zdarzeń:
\begin{inparaenum}[\itshape 1\upshape.]
  \item \f{mouseover} przez metodę \f{magnify()},
  \item \f{mousemove} przez metodę \f{resize()} oraz
  \item \f{mouseout} przez metodę \f{shrink()}.
\end{inparaenum}
\begin{description}
  \item[Metody \f{magnify()} i \f{shrink()}] \hfill \\
    Metody \f{magnify()} i \f{shrink()} nie wpływają bezpośrednio na rozmiar
    ikon.  Modyfikują tylko globalną zmienną \f{factor}, która przechowuje stan
    zaawansowania powiększenia w przypadku odpowiednio najechania/zjechania
    kursora na/z menu.
    Metoda \f{magnify()} przerywa animację zmniejszania, i następnie
    inkrementuje zmienną \f{factor}, do momentu gdy \f{factor == 1}.  Przy
    każdym zwiększeniu, wywoływana jest metoda \f{redraw()}, co przy krótkich
    odstępach wywołań daje wrażenie płynnego zwiększania ikony. Metoda
    \f{shrink()} działa analogicznie.
  \item[Metoda \f{resize()}] \hfill \\
    Rozmiar ikon jest kontrolowany w funkcji \f{resize()}. Funkcja przy każdym
    poruszeniu kursorem nad menu wylicza nowe, żądane wielkości ikon. Do
    wyliczenia wykorzystuje aktualne położenie kursora pobierane z obiektu
    zdarzenia --- \f{event}, zasięg działania powiększenia --- \f{range}
    ustalony dowolnie, oraz skalę powiększenia --- \f{scale} też ustalaną
    ręcznie.
  \item[Metoda \f{redraw()}] \hfill \\
    W każdym wywołaniu powyższych funkcji, jest wywoływana metoda \f{redraw()},
    która modyfikuje styl elementów bezpośrednio w dokumencie. Zmianie ulegają
    wysokość i szerokość każdego elementu, jego margines górny, by ikony były
    wyrównane do dołu oraz całkowita szerokość listy, by zmieścić ikony oraz
    zachować jej wyśrodkowanie.
    Zapisywana wartość wysokości i szerokości jest wypadkową dwóch czynników;
    maksymalnego rozmiaru ikony wyliczonego w \f{resize()}, oraz zaawansowania
    powiększania modyfikowanego odpowiednio przez \f{magnify()} lub
    \f{shrink()}.
\end{description}



\section{Wnioski}

Wykorzystując ścisłe powiązanie \f{JavaScript} z \f{DOM} dokumentów, możliwa
jest ścisła kontrola zawartości i wyglądu dokumentów \f{HTML}.

Pewna obiektowość \f{JavaScript} pozwoliła mi zamknąć logikę animacji w
pojedynczym zasięgu \emph{(scope)}, widocznym tylko dla jednego wywołania
funkcji. Również wewnątrz funkcji, niezależne części logiki są podzielone na
metody. Jednoczesne korzystanie ze skryptowości \f{JavaScript} i oferowanej
przez niego obiektowości są całkiem wygodne.

Pewien problem sprawiło mi pobieranie pozycji kursora względem granic elementu
listy \f{\#fancy-toolbar}. W tym przypadku występowały różnice w obsługiwanym
przez przeglądarki \f{API}. W \f{Google Chrome 16.0} i \f{Firefox 8.0} ta sama
funkcja zwracała różne wartości. Ponieważ nie znalazłem dobrej dokumentacji
\f{JavaScript}, metodą prób i błędów napisałem inny sposób znajdowana położenia
kursora.



\section{Uwagi}

Animacja została przetestowana w \f{Google Chrome 16.0}, \f{Firefox 8.0} oraz
\f{Opera 11.52}. Animacja nie działa pod żadną wersją \f{Internet Explorera}. 

Kod \f{HTML} jest zgodny z \f{XHTML 4.01 Strict}. Podstawowe style CSS zawierają
się w specyfikacji \f{CSS 2.1}. Część stylów zawartych w pliku
\href{https://github.com/student-tomasz/pi-cwiczenie-2/tree/b3f54b187e99c161f7a6780524f504c9fd1b00dd/stylesheets/fancy-toolbar.css}{\f{stylesheets/fancy-toolbar.css}}
to deklaracje zawarte w \f{CSS 3}. Ich usunięcie nie wpływa na funkcjonalność
strony, ani na animację ikon oraz pozwala uzyskać pozytywną wynik walidacji.

Strona korzysta z \f{jQuery} do wstawienia adresów pod ikonami walidacji oraz do
wypisania źródeł wykorzystywanych plików. Wyświetlanie źródeł odbywa się przez
wysłanie zapytania \f{AJAX} do serwera po zawartość pliku tekstowego.
Przeglądając stronę lokalnie, wyświetlanie źródeł nie będzie działać.



\end{document}
